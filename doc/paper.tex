% Juneki Hong and Michael Tango
% Declarative Methods
% Professor Eisner
% May 3, 2013


\documentclass{article}
\usepackage{acl2012}
\usepackage{times}
\usepackage{latexsym}
\usepackage{amsmath}
\usepackage{multirow}
\usepackage{url}
\setlength\titlebox{6.5cm}    % Expanding the titlebox


\title{Declarative Methods Term Project: \\ JumanG}
\author{Juneki Hong and Michael Tango}

%\usepackage[ampersand]{easylist}

\date{}
\begin{document}
\maketitle

\begin{abstract}
We attempted to display a description of a graph in a nicely formatted 
We approached this problem using a variety of different algorithms
\end{abstract}

\section{Introduction}

\section{Graph Drawing}


\section{Declarative Pipeline}

We had an Encoder, Solver, and a Decoder

\subsection{The Encoder}
The Dot Parser

\subsection{The Solvers}

\subsubsection{Radial Solver}
Radial displays have been used for some time to display trees of various data in a way different from the typical top-down approach.
It is unusual, however, to use it to display graphs. We wrote our solver to evenly distribute nodes around the root at each level,
rather than using the traditional style of weighting them based on the number of subnodes as is typical for a solver like this. 
On graphs that are smaller than 

\subsubsection{Springs Solver}
citation for the springs\cite{springs}

\subsubsection{Layered Solver}
Layered graph drawing is an approach to arranging directed acyclic graphs.


\subsection{The Decoder}
Tikz


\section{Experimental Comparison}


\section{Future Work}


\section{Conclusions}

\bibliographystyle{plain}
\bibliography{citations}




\end{document}
